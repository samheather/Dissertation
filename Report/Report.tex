% to cite stuff use \cite{cite_key}

% Heirarchy:
%\part{part}	-1
%\chapter{chapter}	0
%\section{section}	1
%\subsection{subsection}	2
%\subsubsection{subsubsection}	3
%\paragraph{paragraph}	4
%\subparagraph{subparagraph}	5

\documentclass{article}
\usepackage{times}
\usepackage{cite}
\usepackage[autostyle]{csquotes}
\usepackage{marginnote}

%\textwidth = 490pt

\usepackage{graphicx}
\DeclareGraphicsExtensions{.pdf,.png,.jpg}
\graphicspath{ {./figures/} }

\begin{document}

\title{Bringing Knowledge Through AI and SMS}
\author{Sam Heather\\
  Department of Computer Science,\\
  The University of York,\\
  \texttt{sam@heather.sh}}
\date{\today}
\maketitle

\newpage

\begin{abstract}

\marginnote{It is early days so it is normal that many things are missing from the abstract. You should keep in mind that it should include later on few sentences on method/experiment, result and conclusion.}

In remote Africa, there are millions of disadvantaged and  uneducated individuals, who, in the vast-majority, do not have access to the internet and the access to knowledge that this brings.  Outside of their immediate friends and family, individuals can not get access to the information they need on anything from their own body to social problems.

In many parts of the world, the number of people without access to the internet, but access to a mobile phone, is significant.  This project aims to research and develop a system capable of bringing knowledge through a question and answer based interactive system, in the language natively spoken by the user, through the use of a simple Artificial Intelligence and an SMS interface.  The system will be expandable, such that it can be adjusted to handle questions on any knowledge area.

This project raises ethical issues relating to the responsibility of providing accurate information when in a position of trust, the ethics of machine translation and maintaining user privacy.
\end{abstract}

\newpage

\tableofcontents

\newpage

\section{Introduction / Pre-existing Literature}
%%%%%\label{sec:sectionReferenceName}

% expand introductio nto explain why sms, then look at related work in next section.  comppare to smrat phones loosely.
A significant amount of work has previously been undertaken in the area of making knowledge quickly accessible to us in the form of questions and answers.  To take one example, consider the situation of suspecting that a relative was suffering from a disease, for example, Cholera.  The initial step would be to search for information before seeking medical assistance.  In the connected world, this is easy - a simple Google search returns the answers.

\begin{figure}[htb] 
\includegraphics[width=\linewidth]{googleCholera}
\caption{Some caption}
\label{fig:cholera}
\end{figure}

% Lilian: You should give a caption to all figures and tables. use \label{fig:WebQuery} after \caption{Web query result for "Symptoms...}  for example. In this instance you should mentioned the date you made the query and which search engine you have used.


This is easy to do in the developed world, where up to 75\% of the population are connected to the internet~\cite{ITU_Cell_Usage_2013} (with the population of elderly people been largely responsible for the remaining 25\%~\cite{Gov_Internet_Usage_UK_2014}).  This contrasts strongly with the situation in the African continent, where, in 2013, internet usage had only reached 16\%~\cite{ITU_Cell_Usage_2013}, a figure largely inflated by South Africa, where 5\% of the African population generate 2/3 of internet traffic from the African Continent. ~\cite{ITU_Cell_Usage_2013}

% But there are other communication methods which are proven to work
\subsection{Previous work using SMS}
Although the above suggests that the African continent is majoritively disconnected, this is not the case.  Because of restrictions in the electricity available, the ways in which a mobile phone can be used in Africa have far surpassed those in the developed world ~\cite{Fox:2011:Online}.  Interesting such examples include automated services which SMS HIV/AIDS sufferers, reminding them of medication to take, and providing access to current market information for agriculture and farming, saving farmers from making daily trips (often many kilometres) or relying on out-of-date information from a weekly radio broadcast ~\cite{Aker_Mobile_Phones_2010}.

Interactive systems have also been developed to operate over SMS.  One successful example includes mobile money platforms.  These allow for users, from any background, to pay and be paid for goods, and to transfer money across long distances, at negligible cost ~\cite{Aker_Mobile_Phones_2010}.  One of the most highly adopted services is M-Pesa - which, in Kenya alone, was responsible for £5.7 Billion in transfers in 2012 \footnote{Data from Safaricom, M-Pesa operator in Kenya - actual value 817,085,000,000 Kenyan Shillings, converted to GBP on 1st November 2014 at rate of approximately 0.007.}.  M-Pesa gives users a balance linked to a national ID number, from which they can pay for goods by sending an SMS with a cashier (recipient) number or pay outstanding bills in a similar way.  Non-subscribers can also use the system, by depositing money with a M-Pesa cashier in exchange for an access code, which can be sent via SMS to a contact, who can subsequently redeem it with their local cashier.~\cite{Aker_Mobile_Phones_2010}

This difference in standard use of cell phones is demonstrated in the International Telecommunications Unions's 2013 report, which shows that in Europe, for 790 million mobile subscriptions, 53\% of subscribers have mobile internet access (422 million), compared to 17\% in Africa (93 million have mobile broadband, out of 545 mobile subscribers).  This is due to the prohibitively high cost of accessing data services, regardless of the hardware that the user has.  \footnote{In 2012 in Europe, 500 MB of data per month for 12 months cost 1.2\% of the average Gross National Income Per Capita (GNI pc).  In Africa, the average price was 30x this, at 36.2\% of an individuals GNI pc.  ~\cite{ITU_Information_Society_2013}}

% No research has been undertaken investigating whether bringing knowledge through this method is viable option.

\subsection{Ethical Issues}
This project raises a swathe of ethical issues related to translation accuracy, providing information to people in an ethical way and maintaining user privacy.

\subsubsection{Ethics of Providing Information}
One significant issue for this project stems from providing information that may affect an individual or lead them to take a harmful action.  In a similar way to that which a teacher has a responsibility to teach accurate information to a pupil, due to their position of trust, any service relied upon by a user must equally provide accurate information.  The result of not doing this could be providing inaccurate information that leads to a user carrying out an action that causes harm.

% Lilian: You may want to look toward news paper (online/printed) to look at current/past issues that led to a law suit/inquiry and subsequent verdict/laws that may have been changed.

% Lilian: You should provide both point of view so do use this material, then you will have to critic both point of views. 
{\bf this needs to be waaaayyyyy expanded, but I struggled to find information that was useful.  There's lots on remote education, and on basic ethics of teaching, but I struggled to find anything specifically relating to ethics of ensuring information you provide is correct (actually I found some material that pointed the other way)}.

\subsubsection{Ethics of Translation}
A significant part of this project is represented by the support of multiple languages.  It is clear that translation on demand, at scale, needs to be automated by some kind of machine or algorithmic translation.

This project involves two blocks of translation.  These are:
\begin{itemize}
  \item The translation of the user's input into the language of the system (English)
  \item The translation of the answer to a users question from the system language to their local language.\ldots
\end{itemize}
The second of these raises some issues.  To understand these, it is necessary to first understand the two categories of machine translation.

Rule-based machine translation effectively treats human language in a similar style to programming languages.  Formal grammars and lexicons are used to represent words that exist in either the source or target language, structures representing the translation of individual words or groups of words.  Map structures map individual or groups of words to their translated counterparts, sometimes with multiple results (a 1-many map), from which rules decide which is selected.  Maps and rule sets are created by trained computational linguists. ~\cite{kenny2011ethics}

More commonly, Statistical Machine Translation (SMT) is used (for example, this is used by Google and Microsoft Bing Translate).  ~\cite{kenny2011ethics, Google_Translate_Research}  Statistical machine translation learns maps between strings of words of potentially non-equal lengths from pre-existing original texts and their trusted human translations.  The accuracy and breadth of language support for translation increases as more source material is analysed by the system, as potentially erroneous or low quality translations can be identified and marked.  SMT is also dependent on the quality of the human translation on the input material ~\cite{kenny2011ethics}.

The aforementioned issue that is present in statistical machine learning systems comes from the knowledge we assume an individual has and derives from a word.  In human translation, this is solved by the translators knowledge of the difference in material culture, allowing them to append necessary information to the resulting translation that the recipient might find useful.  In statistical machine learning, cultural awareness of material knowledge is a separate problem on it's own, on this scale.  {\bf formal reference to this, probably again from Kenny, but should be able to find better}.  To take one example, from Melby (2006),

\blockquote{"when translating a French menu, a human translator might stop to think that an English speaker in France would appreciate being told that a steak tartare is served entirely raw, even if this information is not contained in the original text (because French people might be assumed to know this already). Such a translator would be aware of differences in material culture, and would be able to empathise with the English speaker who might choose to avoid the dish, given more information"  ~\cite{melby2006can} ~\cite{kenny2011ethics}}

% Lilian: Should just reference the original source in the above.

This issue is a result of the translation engine not been capable of taking as input, and using, a complete representation of the expected cultural differences between those who speak the input language and those who speak the output language. ~\cite{melby2006can}

{\bf below should really be expanded and be in the discussion.  Explain the decision to keep translated answers in the database, which are pre-approved, to ensure no cultural misunderstandings.}

Because the potential use of this system, regardless of whether this is within this project or by a third-party using the results of this project upon completion, could include distributing information that may be used by an individual making an important decision, cultural confusion such as this could have potentially catastrophic effects.

\subsubsection{User Privacy and Data Protection}
Privacy is a significant part of this project, since the


% Lilian: This section should be expended, especially mentioning the consequences of such assumptions. Why do you make those assumption (e.g. time constraint, little or no consequences, ...). What limitation this may cause to your findings.

\subsection{Assumptions Within This Project}
This project will make the assumption that users of this service have basic literacy skills in a language supported by the project.  Although this is not true for the whole of Africa, expanding the remit of this project to include a 'graphical' user interface that works over SMS is a challenge larger than would fit in an undergraduate dissertation.

\section{Discussion}
This section's content...

\subsection{Protecting the Identity of Users}
This subsection's content...

\subsubsection{The Purpose of a Non-reversible Hash}
This subsubsection's content...

\section{Extending this project}
If the author was able to expand this project, areas for expansion include:
\begin{itemize}
  \item Addressing the issue of literacy
  \item Automating the expansion of the database \ldots
\end{itemize}

\bibliography{mybib}{}
\bibliographystyle{plain}

\section{Appendix 1 - Data usage}

\end{document}  %End of document.